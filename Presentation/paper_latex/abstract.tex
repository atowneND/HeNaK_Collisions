\documentclass[letterpaper,11pt]{article}
\author{Ashley Towne\\Advisor: A. P. Hickman}
\title{Semiclassical Analysis of Quantum Mechanical Calculations of Rotationally Inelastic Collisions of He and Ar with NaK$^\dagger$}
\pagestyle{plain}
\usepackage{amsmath}
\newcommand{\vectorize}[1]{\boldsymbol{#1}}

\begin{document}
\maketitle
\begin{center}
    \abstract{Recent quantum mechanical calculations and laboratory
        experiments at Lehigh University have provided detailed information
        about rotationally inelastic collisions of He and Ar with NaK in a cell
        at thermal energies.  The purpose of this project was to develop a
        semiclassical model for these collisions based on the well-known
        vector model.  In the quantum mechanical theory, Grawert coefficients $B_\lambda(j,j')$
        (where $\lambda$ is an integer) give the probability that a discrete
        amount $\lambda \hbar$ of angular momentum is transferred from the
        projectile to the target in a transition between rotational levels $j$
        and $j'$.  Derouard showed that one can develop a semiclassical model by
        transforming from $\lambda$ to the continuous variable $\alpha$, the
        angle between initial and final angular momentum vectors
        $\vectorize{j}$ and $\vectorize{j'}$.  In the present work we invoked
        the vector model, which relates the polar angle $\theta$ of the angular
        momentum vector to the azimuthal quantum number $m$, and showed that
        the distribution $P(\theta,\theta')\sin\theta'$ of final polar angles
        $\theta'$ could be expressed as a convolution of the semiclassical
        Grawert coefficient $B(j,j';\cos\alpha$).  Using this expression we
        calculated the expected distribution of values of $\Delta \theta =
        \theta' - \theta$ and compared it with the quantum mechanical result.  The semiclassical model
        agreed very well with the quantum mechanical theory, especially when the quantum number
        $j$ was large.  The distribution of projections of $j$ onto the $z$
        axis before and after collision (in a transition $jm \rightarrow j'm'$)
        demonstrated (as others have also noticed) that $m$ changes in such a
        way that $\theta$ tends to be preserved. The semiclassical model also predicts the
        propensity for collision-induced changes in $j$ to be even numbers, in
    agreement with quantum mechanical theory and experiment.}

%    \vspace{10pt}

    \noindent{$^\dagger$\small{Work supported by NSF grants PHY-1359195, PHY-0968898, and PHY-1403060.}}

\end{center}

\clearpage
\end{document}
